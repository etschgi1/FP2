% class
\documentclass[ngerman]{scrartcl}

% input preamble
\input{../../shared_preamble.tex}

% biblatex
\addbibresource{festkoerperphysik.bib}


% manual header
\ihead{Festkörperphysik}  % inner (left) head
\chead{\textsc{Wachmann} Elias (12004232)\\\textsc{Zach} Andreas (12004790)}  % center head
\ohead{24.03.2023}  % outer (right) head


% source: https://tex.stackexchange.com/a/40628/255344
\newcommand{\mathdirectcurrent}{\mathrel{\mathpalette\mathdirectcurrentinner\relax}}
\newcommand{\mathdirectcurrentinner}[2]{%
  \settowidth{\dimen0}{$#1=$}%
  \vbox to .85ex {\offinterlineskip
    \hbox to \dimen0{\hss\leaders\hrule\hskip.85\dimen0\hss}
    \vskip.35ex
    \hbox to \dimen0{\hss
      \leaders\hrule\hskip.17\dimen0
      \hskip.17\dimen0
      \leaders\hrule\hskip.17\dimen0
      \hskip.17\dimen0
      \leaders\hrule\hskip.17\dimen0
    \hss}
    \vfill
  }%
}
\newcommand{\textdirectcurrent}{\mathdirectcurrentinner{\textstyle}{}}



\begin{document}

\begin{titlepage}
    \centering
    \includegraphics[width=0.5\textwidth]{../../99_Misc/Logo_KF.pdf}\par\vspace{0.8cm}
    {\scshape\LARGE{Karl-Franzens-Universität Graz}\par}
    {\scshape\LARGE{Institut für Physik}\par}
    \vspace{1cm}
    {\scshape\Large{23S PHY.L02UB Fortgeschrittenenpraktikum 2}\par}
    678 Bachelorstudium Physik, UG2002/2021W\par
    \vspace{1.5cm}
    {\huge\bfseries V. Festkörperphysik\par}
    \vspace{2cm}
    \begin{table}[H]
        \centering
        \begin{tabular}{c c c}
            \Large Wachmann Elias &  & \Large Zach Andreas \\
            \Large 12004232       &  & \Large 12004790     \\
            \multicolumn{3}{c}{Gruppe 12}
        \end{tabular}
    \end{table}
    \vfill
    \Large Betreut von\par
    Thomas Georg \textsc{Boné}, BSc MSc
    \vfill
    {\large 24.03.2023\par}
\end{titlepage}

\clearpage
\tableofcontents
\newpage

\section[Aufgabenstellung]{Aufgabenstellung \cite{ref:angabe_elektronen,ref:angabe_esr}}
\label{sec:aufgabenstellung}

Die im vorliegenden Protokoll beschriebene Laboreinheit zum Thema \textit{Festkörperphysik} gliedert sich in die drei folgenden Teilversuche:
%
\begin{itemize}
    \item \textbf{Elektronenbeugung an einer polykristallinen Graphitprobe}
          \begin{itemize}
              \item Berechnung der Wellenlänge der Elektronen in Abhängigkeit der Anodenspannung
              \item Bestimmung des Gitterabstands von Graphit aus den ersten beiden Beugungsringen
          \end{itemize}
    \item \textbf{Elektronen im Magnetfeld}
          \begin{itemize}
              \item Messung der Auslenkung des Elektronenstrahls in Abhängigkeit der Stromstärke bei zwei unterschiedlichen Anodenspannungen
              \item Berechnung des Krümmungsradius der Elektronenbahn aus der Auslenkung
              \item Berechnung der magnetischen Induktion aus dem Spulenstrom
              \item Grafische Darstellung von $\nicefrac{1}{r}$ in Abhängigkeit von $B$
              \item Bestimmung der spezifischen Ladung des Elektrons durch lineare Regression
          \end{itemize}
    \item \textbf{Elektronen-Spin-Resonanz}
          \begin{itemize}
              \item Bestimmung des Resonanzmagnetfeldes $B_0$ in Abhängigkeit von der gewählten Resonanzfrequenz
              \item Bestimmung des Landé-Faktors von 1,1-Diphenyl-2-Pikryl-Hydrazyl (DPPH)
          \end{itemize}

\end{itemize}



\section{Voraussetzungen und Grundlagen}
\label{sec:voraussetzungen_grundlagen}

\subsection[Elektronenbeugung]{Elektronenbeugung \cite{ref:angabe_elektronen}}
\label{subsec:grundlagen_elektronenbeugung}

\paragraph{Berechnung der Elektronenwellenlänge}
%
Um die im Versuch auftretenden Interferenzerscheinungen zu erklären wird den Elektronen, die beim Auftreffen auf die polykristalline Graphitprobe den Impuls $p$ besitzen, eine Wellenlänge $\lambda$ zugeordnet. Dieser Zusammenhang wird mit der Gleichung von de Broglie beschrieben:
%
\begin{equation}
    \label{eq:de_broglie}
    \lambda = \frac{h}{p}
\end{equation}
%
wobei $h=\SI{6.62607015e-34}{Js}$ das Planck'sche Wirkungsquantum beschreibt.

Der Impuls kann aus der Geschwindigkeit $v$ bestimmt werden, die die Elektronen nach Durchlaufen einer Beschleunigungsspannung $U_B$ erreicht haben:
%
\begin{equation}
    \label{eq:kinetische_energie}
    \frac{1}{2} m_e v^2 = \frac{v^2}{2 m} = q_e U_B
\end{equation}
%
Dabei beschreiben $q_e = \SI{1.602e-19}{As}$ die Elementarladung ($\equiv$ negative Elektronenladung) und $m_e = \SI{9.109e-31}{kg}$ die Ruhemasse des Elektrons.

Die Wellenlänge der Elektronen ergibt sich somit zu:
%
\begin{equation}
    \label{eq:wellenlaenge_elektronen}
    \lambda = \frac{h}{\sqrt{2 m_e q_e U_B}}
\end{equation}

\paragraph{Beugung von Elektronen an Kristallgittern}
%
In unserem Versuch trifft ein Elektronenstrahl auf eine polykristalline Graphitprobe und wird gemäß der Bragg-Bedingung gestreut:
%
\begin{equation}
    \label{eq:bragg}
    2 d \sin(\theta) = n \lambda \qq{mit} n \in \mathbb{N}\setminus\{0\}
\end{equation}
%
Dabei ist $d$ der Abstand zwischen den einzelnen Netzebenen im Graphitgitter, $n$ die Beugungsordnung und $\theta$ der Bragg-winkel (Winkel zwischen Elektronenstrahl und Gitterebenen).
%
\begin{figure}[H]
    \centering
    \begin{samepage}
        \includegraphics[width=0.4\linewidth]{fig/bragg-reflexion.pdf}
        \caption{Bragg-Gesetz}
        \label{fig:bragg_reflexion}
    \end{samepage}
\end{figure}
%
In einer polykristallinen Graphitprobe sind die Bindungen zwischen den einzelnen Lagen gebrochen, wodurch ihre Orientierung zufällig ist. D.h. man findet immer wieder
Mikrokristalline mit der richtigen Orientierung zum Elektronenstrahl, sodass die Bragg-Bedingung erfüllt ist. Der gebeugte Elektronenstrahl ist daher in Form eines Konus aufgefächert, wodurch die Interferenzringe am Schirm entstehen.
%
\begin{figure}[H]
    \centering
    \begin{samepage}
        \includegraphics[width=0.6\linewidth]{fig/graphitgitter.png}
        \caption[Graphitkristallgitter]{Kristallgitter von Graphit. Die obere Abbildung zeigt die Gitterebenen von Graphit für die ersten beiden Interferenzringe. Quelle: \cite{ref:angabe_elektronen}}
        \label{fig:graphitgitter}
    \end{samepage}
\end{figure}
%
Der Bragg-Winkel kann aus dem Radius $r$ des am Schirm sichtbaren Interferenzringes berechnet werden, wobei beachtet werden muss, dass der Ablenkungswinkel $\alpha = 2\theta$ doppelt so groß ist. Aus \autoref{fig:bragg_reflexion} sieht man direkt:
%
\begin{equation}
    \label{eq:ablenkwinkel_radius}
    \sin(2\alpha) = \frac{r}{R}
\end{equation}
%
mit $R$, dem Radius der Glaskugel und $r$ ist der Radius des Interferenzringes. Für kleine Winkel $\alpha$ gilt:
%
\[\sin(2\alpha) \approx 2\sin(\alpha)\]
%
Dadurch erhält man für kleine Winkel $\theta$:
%
\[\sin(\alpha)=\sin(2\theta)\approx2\sin(\theta)\]
%
Mit dieser Näherung erhält man:
%
\begin{equation}
    \label{eq:radius_wellenlaenge}
    r=\frac{2R}{d}n\lambda
\end{equation}
%
Die Radien der zwei inneren Interferenzringe stammen von den Netzebenen $d_1$ und $d_2$ des Graphits für $n=1$.


\subsection[Elektronen im Magnetfeld]{Elektronen im Magnetfeld \cite{ref:angabe_elektronen}}
\label{subsec:grundlagen_elektronen_im_magnetfeld}
%
Bewegt sich ein Elektron im Magnetfeld $\vb*{B}$, so wirkt auf dieses die Lorentzkraft:
%
\begin{equation}
    \label{eq:lorentzkraft}
    \vb*{F}_L = - q_e \left(\vb*{v} \times \vb*{B} \right)
\end{equation}
%
Wenn man davon ausgeht, dass Geschwindigkeitsvektor $\vb*{v}$ und der Vektor der magnetischen Flussdichte $\vb*{B}$ senkrecht aufeinander stehen, so gilt:
\begin{equation}
    \label{eq:lorentzkraft_orthogonal}
    F_L = -q_e \cdot v \cdot B
\end{equation}
%
Die Lorentzkraft wirkt als Zentripetalkraft und zwingt das Elektron auf eine Kreisbahn mit dem Radius $r$, sodass folgende Beziehung gilt:
\[-q_e \cdot v \cdot B = \frac{m_e \cdot v^2}{r}\]
Die spezifische Ladung ergibt sich dann zu:
\begin{equation}
    \label{eq:spezifische_ladung}
    q_{\text{spez}} = - \frac{q_e}{m_e} = \frac{v}{rB}
\end{equation}
Die kinetische Energie gewinnt das Elektron aus einem elektrischen Feld, in dem es beschleunigt wird (\autoref{eq:kinetische_energie}). Aus den beiden Gleichungen folgt nun:
\begin{equation}
    \label{eq:spezifische_ladung_U}
    q_{\text{spez}} = = \frac{2 U_A}{(rB)^2}
\end{equation}
Der Radius $r$ der Kreisbahn lässt sich nicht direkt messen. Stattdessen wird diese Größe aus der Ablenkung $s$ im Magnetfeld und aus dem Kolbendurchmesser $d$ ermittelt. Man betrachte dazu \autoref{fig:kolben_skizze}.

% \centering
\setcapindent{0pt}
\begin{minipage}[t]{0.35\linewidth}
    \begin{figure}[H]
        \centering
        \includegraphics[width=\linewidth]{fig/Kolben_skizze.png}
        \caption[Kolben Skizze]{Schematische Skizze des Kolbens}
        \label{fig:kolben_skizze}
    \end{figure}
\end{minipage}%
\hspace*{\fill}
\begin{minipage}[t]{0.6\linewidth}
    \vspace{1cm}
    $d$ = 135 mm\\ \\
    Die magnetische Flussdichte $\vb*{B}$ ist:\\
    $\vb*{B}$ = $\mu_0$ $\vb*{H}$ (T = Tesla)\\
    $\mu_0$ = 4$\pi$ $\cdot$ 10$^{-7}$ \si{\henry\per\meter} (Permeabilitätskonstante)\\
    $\vb*{H}$ -- die Feldstärke\\ \\
    Die Feldstärke des nahezu homogenen Magnetfeldes des Helmholtz-Spulenpaars ist: \\
    \begin{equation}
        {H} = \frac{{B}}{\mu_0} = \frac{n R^2 I}{\left(R^2 + a^2\right)^{\nicefrac{3}{2}}} = \SI{33.8e2}{\ampere\per\meter}
    \end{equation}
    $n$ -- Windungszahl je Spule ($n = 320$)\\
    $R$ -- Radius der Spulen ($R = \SI{6.8}{\centi\meter}$)\\
    $a$ -- halber Spulenabstand ($a = \SI{3.4}{\centi\meter}$)\\
    $I$ -- Stromstärke je Spule\\ \\
\end{minipage}
\setcaphanging


\subsection[Elektronen-Spin-Resonanz]{Elektronen-Spin-Resonanz \cite{ref:angabe_esr}}
\label{subsec:grundlagen_elektronenspinresonanz}
%
Seit ihrer Entdeckung durch E. K. Zavoisky (1945) hat sich die Elektronenspinresonanz (ESR) zu einer wichtigen Methode zur Untersuchung von Molekül- und Kristallstrukturen, von chemischen Reaktionen und anderen Problemstellungen in Physik, Chemie, Biologie und Medizin entwickelt. Sie beruht auf der Absorption hochfrequenter elektromagnetischer Strahlung durch paramagnetische Stoffe in einem äußeren Magnetfeld, in dem die Spinzustände der Elektronen aufspalten.

Elektronen-Spin-Resonanz ist auf paramagnetische Stoffe begrenzt, da hier die Bahndrehimpulse und Spins der Elektronen zu einem von Null verschiedenen Gesamtdrehimpuls koppeln. Geeignet sind z. B. Verbindungen, in denen Atome mit nicht vollständig aufgefüllten inneren Schalen (Übergangsmetalle, seltene Erden) eingebaut sind, organische Moleküle (freie Radikale), die einzelne ungepaarte Elektronen enthalten, oder Kristalle mit Gitterfehlstellen in einem paramagnetischen Zustand.

Mit dem Gesamtdrehimpuls $\vb*{J}$ verbunden ist das magnetische Moment
\begin{equation}
    \vb*{\mu}_J = g_J \cdot \frac{\mu_B}{\hbar}\cdot \vb*{J}
\end{equation}
In einem Magnetfeld $\vb*{B}_0$ erhält das magnetische Moment $\vb*{\mu}_J$ die potentielle Energie:
\begin{equation}
    E = - \vb*{\mu}_J \cdot \vb*{B}_0
\end{equation}
Sie ist gequantelt, da das magnetische Moment und der Gesamtdrehimpuls nur bestimmte diskrete Orientierungen zum Magnetfeld einnehmen können. Jeder Orientierung des Drehimpulses entspricht ein Zustand bestimmter potentieller Energie im Magnetfeld. Für die Komponente $J_z$ des Gesamtdrehimpulses parallel zum Magnetfeld gilt
\begin{equation}
    J_z = \hbar \cdot m_J \qq{mit} m_J = -J, -\left(J-1\right), \dots, J-1, J
\end{equation}
wobei die Drehimpulsquantenzahl $J$ ganz- oder halbzahlig ist, d.h. die potentielle Energie spaltet auf in die diskreten Zeeman-Niveaus.
\begin{equation}
    E = g_J \cdot \mu_B \cdot B_0 \cdot m_J \qq{mit} m_J = -J, -\left(J-1\right), \dots, J-1, J
\end{equation}
Mit der Methode der Elektronen-Spin-Resonanz kann die Energieaufspaltung direkt gemessen werden. Dazu wird senkrecht zum statischen Magnetfeld $\vb*{B}_0$ ein hochfrequentes magnetisches Wechselfeld
\begin{equation}
    \vb*{B}_1 = \vb*{B}_{\text{HF}} \cdot \sin\left(\omega t\right) = \vb*{B}_{\text{HF}} \cdot \sin\left(2 \pi \nu t\right)
\end{equation}
in die Probe eingestrahlt. Wenn die eingestrahlte Energie $h \cdot \nu$ dem Energieabstand $\Delta E$ zweier benachbarter Energieniveaus entspricht, d.h. wenn die Bedingungen
\begin{equation}
    \label{eq:V}
    \Delta m_J = \pm 1 \qq{und} \Delta E = h \cdot \nu = g_J \cdot \mu_B \cdot B_0
\end{equation}
erfüllt sind, bewirkt das Wechselfeld ein \enquote{Umklappen} der magnetischen Momente von einer Einstellmöglichkeit im Magnetfeld $B_0$ in die andere. Anders ausgedrückt werden Übergänge zwischen den benachbarten Niveaus induziert und es tritt ein Resonanzeffekt auf, der sich in der Absorption von Energie aus dem eingestrahlten magnetischen Wechselfeld durch die Probe zeigt.
%
\begin{figure}[H]
    \centering
    \begin{samepage}
        \includegraphics[width=0.65\linewidth]{fig/zeeman_grundlage.png}
        \caption[Zeeman-Effekt]{Energieaufspaltung des freien Elektrons im Magnetfeld
            und Resonanzbedingung für die Elektronen-Spin-Resonanz.}
        \label{fig:grundlage_zeeman}
    \end{samepage}
\end{figure}
%
In zahlreichen Verbindungen spielt der Bahndrehimpuls eine untergeordnete Rolle und die Betrachtungen können sich auf den Spin der Elektronen beschränken. In \autoref{fig:grundlage_zeeman} ist die Situation der Einfachheit halber für ein freies Elektron dargestellt: Hier besteht der Gesamtdrehimpuls aus dem Spin $\vb*{s}$ des Elektrons.

Die Drehimpulsquantenzahl ist
\begin{equation*}
    J = s = \frac{1}{2}
\end{equation*}
und der Landé'sche Faktor hat den Wert
\begin{equation*}
    g_J = g_s \approx 2,0023
\end{equation*}
Im Magnetfeld spaltet die Energie des Elektrons auf in die beiden Niveaus
\begin{equation}
    E = g_S \cdot \mu_B \cdot B_0 \cdot m_s \qq{mit} m_S = \pm \frac{1}{2}
\end{equation}
die einer Ausrichtung des Elektronenspins antiparallel und parallel zum Magnetfeld entsprechen. Bei einem Übergang zwischen den beiden Niveaus ist die Auswahlregel \ref{eq:V} automatisch erfüllt: Die Resonanzbedingung lautet analog zu \ref{eq:V}
\begin{equation}
    h \cdot \nu = g_S \cdot \mu_B \cdot B_0
\end{equation}
Misst man nun z.B. die aus dem eingestrahlten Wechselfeld absorbierte Energie bei fester Frequenz $\nu$ in Abhängigkeit vom Magnetfeld $B_0$, so erhält man eine Absorptionslinie mit einer Halbwertsbreite $\delta B_0$. Diese Linienbreite ist bei einem homogenen Magnetfeld im einfachsten Fall Ausdruck der Unschärfe $\delta E$ des Übergangs. Es gilt die Unschärferelation
\begin{equation}
    \delta E \cdot T \geq \frac{h}{2},
\end{equation}
wobei $T$ die Lebensdauer des Niveaus ist. Wegen \autoref{eq:V} gilt
\begin{equation}
    \delta E = g \cdot \mu_B \cdot \delta B_0
\end{equation}
damit ist unabhängig von der Resonanzfrequenz $\nu$
\begin{equation}
    \delta B_0 = \frac{\hbar}{2 \cdot g_J \cdot \mu_B \cdot T}
\end{equation}
Ausgewertet wird die Lage und die Breite der Absorptionslinien im ESR-Spektrum der untersuchten Proben:

Aus der Lage bestimmt man gemäß \autoref{eq:V} den Landé'schen Faktor $g_J \approx r_{\text{Probe}}$. Er liegt bei einem freien Atom oder Ion zwischen dem Wert $g_J = 1$, der bei reinem Bahnmagnetismus erreicht wird, und $g_J \approx \num{2.0023}$ für reinen Spinmagnetismus. Tatsächlich sind die mit der Methode der Elektronen-Spin-Resonanz untersuchten paramagnetischen Zentren jedoch nicht frei. In ein Kristallgitter eingebaut oder in Lösung von einer Solvathülle umgeben, wirken starke elektrische und magnetische Felder auf sie, die von den Atomen der Umgebung erzeugt werden. Diese Felder bewirken eine Energieverschiebung und beeinflussen die \textit{Zeeman}-Aufspaltung der Elektronen. Dadurch ändert sich der Wert des g-Faktors, er wird häufig anisotrop, und in den ESR-Spektren tritt eine Feinstruktur auf. Es lassen sich also aus dem g-Faktor Rückschlüsse
auf die Bindungsverhältnisse der Elektronen und den chemischen Aufbau der untersuchten Probe ziehen.

Aus der Linienbreite können Aussagen über die dynamischen Eigenschaften gewonnen werden. Die Linienbreite wird -- wenn man von nicht aufgelösten Feinstrukturen absieht -- bestimmt durch mehrere Prozesse, die der Ausrichtung der magnetischen Momente entgegenwirken. Als Spin-Spin-Relaxation bezeichnet man die Wechselwirkung der ausgerichteten magnetischen Momente untereinander und als Spin-Gitter-Relaxation die Wechselwirkung der magnetischen Momente mit den fluktuierenden elektrischen und magnetischen Feldern, die in Festkörpern durch die Gitterschwingungen oder in Flüssigkeiten durch die thermische Bewegung der Atome erzeugt werden. In einigen Fällen wird die Linienbreite durch sogenannte Austauschwechselwirkung beeinflusst und fällt erheblich kleiner aus, als bei reiner Dipol-Dipol-Wechselwirkung der Spins zu erwarten wäre.

Für praktische Anwendungen entwickelte ESR-Spektrometer arbeiten hauptsächlich bei Frequenzen von etwa $\SI{10}{\giga\hertz}$ (Mikrowellen, X-Band). Die Magnetfelder liegen entsprechend in der Größenordnung $\SIrange{0.1}{1}{\tesla}$. Im vorliegenden Versuch ist das Magnetfeld $B_0$ deutlich schwächer. Es wird mit Hilfe zweier Helmholtz-Spulen erzeugt und kann durch geeignete Wahl des Spulenstromes zwischen $\SIrange{0}{4}{\milli\tesla}$ eingestellt werden. Dem konstanten Spulenstrom wird ein mit $\SI{50}{\hertz}$ modulierter Strom überlagert. Also setzt sich das entsprechend modulierte Magnetfeld B zusammen aus einem Gleichfeld $B_0$ und einem $\SI{50}{\hertz}$-Feld $B_{\text{mod}}$. Die Probe befindet sich in einer HF-Spule, die Teil eines elektrischen Schwingkreises hoher Güte ist. Der Schwingkreis wird durch einen frequenzvariablen HF-Oszillator zwischen $\SIrange{15}{130}{\mega\hertz}$ erregt.

Ist die Resonanzbedingung \ref{eq:V} erfüllt, absorbiert die Probe Energie und der Schwingkreis wird belastet. Als Folge ändert sich der Wechselstromwiderstand des Schwingkreises und die Spannung über der Spule nimmt ab. Diese Spannung wird durch Gleichrichtung und Verstärkung in das Messsignal gewandelt.
%
\begin{figure}[H]
    \centering
    \begin{tabular}{cc}
        \includegraphics[width=0.48\textwidth]{fig/grund_a.png} &
        \includegraphics[width=0.48\textwidth]{fig/grund_b.png}     \\
        a                                                       & b \\
        \includegraphics[width=0.48\textwidth]{fig/grund_c.png} &
        \includegraphics[width=0.48\textwidth]{fig/grund_d.png}     \\
        c                                                       & d \\
    \end{tabular}
    \caption[Oszilloskopbild des Messsignals]{Oszilloskopbild des Messsignals (Y bzw. I) und des modulierten Magnetfeldes (X bzw. II). Links: Zweikanaldarstellung mit DC-gekoppeltem Kanal II. Rechts: XY-Darstellung mit AC-gekoppeltem Kanal II\newline (a) unkorrigierte Phasenverschiebung $\varphi$, zu niedriges Gleichfeld $B_0$\newline (b) korrigierte Phasenverschiebung $\varphi$, zu niedriges Gleichfeld $B_0$\newline (c) unkorrigierte Phasenverschiebung $\varphi$, passendes Gleichfeld $B_0$\newline (d) korrigierte Phasenverschiebung $\varphi$, passendes Gleichfeld $B_0$}
    \label{fig:grundlagen_oszibilder}
\end{figure}
%
Das Messsignal erscheint am Ausgang des Betriebsgerätes zeitverzögert gegenüber dem modulierten Magnetfeld. Die Zeitverzögerung kann als Phasenverschiebung im Betriebsgerät korrigiert werden. Ein Zweikanal-Oszilloskop stellt im X-Y-Betrieb das Messsignal zusammen mit einer zum Magnetfeld proportionalen Spannung als Resonanzsignal dar. Das Resonanzsignal ist symmetrisch, wenn das Gleichfeld $B_0$ die Resonanzbedingung erfüllt und die Phasenverschiebung $\varphi$ zwischen Messsignal und moduliertem Magnetfeld korrigiert ist (siehe \autoref{fig:grundlagen_oszibilder}).

Als Probensubstanz wird 1,1-Diphenyl-2-Pikryl-Hydrazyl (DPPH) verwendet. Diese organische Verbindung ist ein relativ stabiles freies Radikal, das an einem Atom der Stickstoffbrücke ein ungepaartes Valenzelektron aufweist (siehe \autoref{fig:dpph}). Dessen Bahnbewegung ist durch den Molekülaufbau praktisch ausgelöscht. Daher hat das Molekül einen g-Faktor, der fast dem eines freien Elektrons entspricht. Die Substanz ist in polykristalliner Form für die Demonstration der Elektronenspinresonanz sehr gut geeignet, da sie eine intensive ESR-Linie hat, die wegen der Austauschverschmälerung eine geringe Breite aufweist.
%
\begin{figure}[H]
    \centering
    \begin{samepage}
        \includegraphics[width=0.5\linewidth]{fig/dpph.png}
        \caption[DPPH]{Chemische Struktur von 1,1-Diphenyl-2-Pikryl-Hydrazyl (DPPH)}
        \label{fig:dpph}
    \end{samepage}
\end{figure}


\subsection{Unsicherheitsanalyse}
\label{subsec:unsicherheitsanalyse}

Die explizit angegebenen Unsicherheiten der ermittelten Messgrößen basieren auf Berechnungen durch die Unsicherheitsangabe nach den Datenblättern der verwendeten Messgeräte. Diese sind in \autoref{tab:geraeteliste} vermerkt beziehungsweise referenziert.

Die Fehlerfortpflanzung der berechneten Werte basiert auf der Größtunsicherheitsmethode nach Gauß. Um diese Berechnungen zeiteffizient durchführen zu können, wird für jeden Unterpunkt der Laborübung ein Skript in \verb!Python! implementiert. Kernstück dessen ist das package \verb!uncertainties! \cite{ref:uncertainties}, das intern die Fehlerfortpflanzung berechnet. Gerundet wird nach den Angaben des Skriptums der Lehrveranstaltung \enquote{Einführung in die physikalischen Messmethoden} \cite{ref:messmethoden}.



\section{Versuchsanordnung}
\label{sec:versuchsanordnung}

\subsection{Elektronenbeugung und Elektronen im Magnetfeld}
\label{subsec:anordnung_elektronen}

Der Aufbau zu den beiden Teilversuchen \textit{Elektronenbeugung} und \textit{Elektronen im Magnetfeld} ist ident und in \autoref{fig:aufbau_elektronen} dargestellt.
%
\begin{figure}[H]
    \centering
    \begin{samepage}
        \includegraphics[width=0.8\linewidth]{fig/Beugung_Aufbau_beschriftet.png}
        \caption[Aufbau Elektronenbeugung und Elektronen im Magnetfeld]{Versuchsaufbau der beiden Teilversuche \textit{Elektronenbeugung} und \textit{Elektronen im Magnetfeld}. \textbf{(a)} Glühkathode mit Graphitkristallgitter, \textbf{(b)} Elektronenbeugungsröhre, \textbf{(c)} Helmholtz-Spulenpaar, \textbf{(d)} Netzgerät, \textbf{(e)} Hochspannungsnetzgerät, \textbf{(f)} Digitalmultimeter}
        \label{fig:aufbau_elektronen}
    \end{samepage}
\end{figure}



\subsection{Elektronen-Spin-Resonanz}
\label{subsec:anordnung_esr}

Der Aufbau des Teilversuchs \textit{Elektronen-Spin-Resonanz} ist in \autoref{fig:aufbau_esr} dargestellt.
%
\begin{figure}[H]
    \centering
    \begin{samepage}
        \includegraphics[width=0.8\linewidth]{fig/ESR_Aufbau_bearbeitet.png}
        \caption[Aufbau Elektronen-Spin-Resonanz]{Versuchsaufbau zum Teilversuch \textit{Elektronen-Spin-Resonanz}. \textbf{(a)} ESR-Grundgerät, \textbf{(b)} Helmholtz-Spulenpaar, \textbf{(c)} Steckspulen, \textbf{(d)} DPPH-Probe, \textbf{(e)} ESR-Betriebsgerät, \textbf{(f)} Oszilloskop}
        \label{fig:aufbau_esr}
    \end{samepage}
\end{figure}



\section{Geräteliste}
\label{sec:geraeteliste}

Für den praktischen Aufbau und die Messungen der geforderten Größen wurden die in \autoref{tab:geraeteliste} aufgelisteten Geräte und Hilfsmittel verwendet.
%
\begin{table}[H]
    \centering
    \begin{samepage}
        \caption[Geräteliste]{Verwendete Geräte und wichtige Materialien}
        \label{tab:geraeteliste}
        \begin{tblrx}{}
            Gerät                   & Hersteller & Modell   & Inventar-Nr. & Anmerkung                              \\
            Glühkathode             & -          & -        & -            & -                                      \\
            Graphitkristallgitter   & -          & -        & -            & -                                      \\
            Elektronenbeugungsröhre & -          & -        & -            & -                                      \\
            Helmholtz-Spulenpaar    & Leybold    & 555 604  & -            & 320 Windungen, max. \SI{2}{A}          \\
            Netzgerät               & Phywe      & 11704.93 & 003576       & \SI{220}{V~}/\SI{50}{Hz}, \SI{220}{VA} \\
            Hochspannungsnetzgerät  & Leybold    & 522 37   & -            & -                                      \\
            Digitalmultimeter       & TTi        & 1604     & -            & -                                      \\
            ESR-Grundgerät          & Leybold    & 514 55   & 300025940001 & -                                      \\
            Steckspulen             & -          & -        & -            & klein, mittel, groß                    \\
            DPPH-Probe              & -          & -        & -            & -                                      \\
            ESR-Betriebsgerät       & Leybold    & 514 571  & -            & -                                      \\
            Oszilloskop             & HAMEG GmbH & HM205-2  & -            & analog                                 \\
        \end{tblrx}
    \end{samepage}
\end{table}



\section{Versuchsdurchführung und Messergebnisse}
\label{sec:versuchsdurchfuehrung_messergebnisse}

\subsection{Elektronenbeugung}
\label{subsec:durchfuehrung_elektronenbeugung}

Der Aufbau erfolgt wie in \autoref{subsec:anordnung_elektronen} gezeigt. Zu Beginn wird an der Glühkathode über das Hochspannungsnetzgerät eine initiale Beschleunigungsspannung von $U_B = \SI{2}{kV}$ angelegt. Auf der Kugelkalotte am milchtrüben Ende der Elektronenbeugungsröhre erscheinen konzentrische grüne Kreise um einen hellen grünen Punkt im Zentrum. Diese Interferenzmaxima (Punkt sowie Ringe) unterliegen der Bragg-Bedingung (\autoref{eq:bragg}). Mithilfe eines Messschiebers wird nun der Abstand (Radius) vom ersten und zweiten Interferenzmaximum (Ring) zum nullten Maximum (Zentrum) gemessen und zur angelegten Hochspannung notiert. Dieser Vorgang wird weitere viermal wiederholt, wobei die Beschleunigungsspannung jeweils um \SI{0.5}{kV} erhöht wird, dies ist die erste Messserie. Um eine minimale statistische Aussage tätigen zu können, wird diese Messserie ein zweites Mal wiederholt. Die erhaltenen Messwerte finden sich in \autoref{tab:messergebnisse_beugung}.
%
\begin{table}[H]
    \centering
    \begin{samepage}
        \caption[Messergebnisse Elektronenbeugung]{Radien des ersten und zweiten Interferenzmaximumrings $r_{1,i,j}$ und $r_{2,i,j}$ in \unit{mm} in Abhängigkeit von der angelegten Beschleunigungsspannung $U_B$ in \unit{kV}. Laufindex $i$ bezeichnet dabei den Index der Messung (korreliert mit der Spannung), Index $j=\{1\mathcomma2\}$ bezeichnet den Index der Messserie. Unsicherheiten $\Delta r = \SI{2}{mm}$, $\Delta U_B = \SI{0.02}{kV}$.}
        \label{tab:messergebnisse_beugung}
        \begin{tblr}{colspec={S[table-format=1.1]S[table-format=2]S[table-format=2]S[table-format=2]S[table-format=2]}, row{1}={guard}}
            $U_B$ / \unit{kV} & $r_{1,i,1}$ / \unit{mm} & $r_{1,i,2}$ / \unit{mm} & $r_{2,i,1}$ / \unit{mm} & $r_{2,i,2}$ / \unit{mm} \\
            2.0               & 17                      & 17                      & 31                      & 30                      \\
            2.5               & 15                      & 15                      & 27                      & 26                      \\
            3.0               & 13                      & 13                      & 24                      & 24                      \\
            3.5               & 12                      & 13                      & 21                      & 24                      \\
            4.0               & 11                      & 12                      & 20                      & 22                      \\
        \end{tblr}
    \end{samepage}
\end{table}


\subsection{Elektronen im Magnetfeld}
\label{subsec:durchfuehrung_elektronen_b_feld}

Der Aufbau zum Teilversuch zu Elektronen im Magnetfeld ist ident zum vorhergegangenen aus \autoref{subsec:durchfuehrung_elektronenbeugung} und wird nicht verändert. Nun wird zusätzlich zum Hochspannungsnetzgerät auch das allgemeine Netzgerät und das Digitalmultimeter in Betrieb genommen. Schickt man über das allgemeine Messgerät Gleichstrom
durch das Helmholtz-Spulenpaar, so wird zwischen den Spulen ein Magnetfeld erzeugt, über welches die Elektronen die Lorentzkraft erfahren. Dies geschieht natürlich nur bei angelegter Hochspannung an der Glühkathode. Es werden zwei Spannungen gewählt (\SI{1.8}{kV} und \SI{4}{kV}), für die die nachfolgend beschriebenen Messserien durchgeführt werden. Die Stromstärke, welche in den Spulen herrscht, wird vom Digitalmultimeter gemessen. Es werden nun in \SI{5}{mm}-Schritten jene Stromstärken notiert, die nötig sind, um das Zentrum (0. Intensitätsmaximum) des Elektronenstrahls um die aktuelle Distanz abzulenken. Dabei wird bei der Ausgangsstromstärke ($I_0 \approx \SI{0.5}{mA}$) der auf die entsprechende Distanz eingestellte und fixierte Messschieber am Zentrum des Elektronenstrahls auf der Kugelkalotte angehalten und die Stromstärke so lange langsam erhöht, bis das Zentrum das Ende des eingestellten Messschiebers erreicht. Diese Stromstärke wird zur aktuellen Distanz notiert. Für eine geringe statistische Aussagefähigkeit wird diese Messserie ein zweites Mal wiederholt. Die erhaltenen Messwerte finden sich in \autoref{tab:messergebnisse_b_feld}.
%
\begin{table}[H]
    \centering
    \begin{samepage}
        \caption[Messergebnisse Elektronenstrahl im Magnetfeld]{Messergebnisse der vier Messserien zur Auswirkung der Lorentzkraft auf den Elektronenstrahl.
            Gemessen wurde die benötigte Stromstärke $I_{i,j,k}$ zur Ablenkung des Elektronenstrahls um eine bestimmte Distanz $d$ bei gegebener angelegter Hochspannung an der Glühkathode. Index $i$ beschreibt die angelegte Hochspannung, Index $j$ die gegebene Distanz und Index $k$ die erste oder zweite Messserie. Die Messungen wurden je für die Hochspannungen $U_{\text{HV}}=\{\SI{1.8}{kV}\mathcomma\SI{4}{kV}\}$ zweimal durchgeführt. Unsicherheiten $\Delta U_{\text{HV}} = \SI{0.02}{kV}$, $\Delta I = \SI{3}{mA}$, $\Delta d = \SI{2}{mm}$.}
        \label{tab:messergebnisse_b_feld}
        \begin{tblr}{colspec={S[table-format=1.1]S[table-format=2]S[table-format=3]S[table-format=3]}, row{1}={guard}}
            $U_{\text{HV}}$ / \unit{kV} & $d$ / \unit{mm} & $I_{i,j,1}$ / \unit{mA} & $I_{i,j,2}$ / \unit{mA} \\
            1.8                         & 5               & 40                      & 49                      \\
            1.8                         & 10              & 75                      & 78                      \\
            1.8                         & 15              & 100                     & 108                     \\
            1.8                         & 20              & 131                     & 149                     \\
            1.8                         & 25              & 164                     & 180                     \\
            1.8                         & 30              & 195                     & 208                     \\
            1.8                         & 35              & 228                     & 240                     \\
            4.0                         & 5               & 76                      & 60                      \\
            4.0                         & 10              & 122                     & 100                     \\
            4.0                         & 15              & 165                     & 139                     \\
            4.0                         & 20              & 216                     & 190                     \\
            4.0                         & 25              & 261                     & 245                     \\
            4.0                         & 30              & 314                     & 290                     \\
            4.0                         & 35              & 360                     & 340                     \\
        \end{tblr}
    \end{samepage}
\end{table}


\subsection{Elektronen-Spin-Resonanz}
\label{subsec:durchfuehrung_esr}

Der Aufbau erfolgt wie in \autoref{subsec:anordnung_esr} dargestellt. Initial wird die kleinste Steckspule -- jene mit geringster Windungszahl -- gewählt. Diese wird in die dafür vorgesehenen Kontakte am ESR-Grundgerät gesteckt. Die DPPH-Probe wird mittig in die Steckspule eingeführt. Am ESR-Betriebsgerät wird eine Frequenz eingestellt, bei welcher die Probe resoniert. Dies resultiert in einem der in \autoref{fig:grundlagen_oszibilder} dargestellten Bilder am Oszilloskop. Die Phasenverschiebung und die Amplitude der Modulationsspannung werden am ESR-Betriebsgerät derart angepasst, dass sie dem Unterbild (d) aus \autoref{fig:grundlagen_oszibilder} gleichen. Es wird darauf geachtet, dass sich das dargestellte Minimum genau horizontal zentriert im Bildraster des Oszilloskops befindet. Begonnen wird mit einer Resonanzfrequenz von \SI{120}{MHz}, diese wird nun in Zehnerschritten reduziert (bis \SI{20}{MHz}) und die jeweilige Gleichstromstärke, die nötig ist, um das eben beschriebene Bild am Oszilloskop zu erhalten, notiert. Diese wird direkt vom ESR-Betriebsgerät abgelesen. Gelangt man unter den Frequenzbereich der verwendeten Spule, in welchem noch Resonanzeffekte auftreten, so wird die Spule durch eine größere (mehr Windungen) ausgetauscht. Es stehen drei unterschiedliche Steckspulengrößen zur Auswahl. Die zur verwendeten Steckspule und eingestellten Resonanzfrequenz abgelesene Gleichstromstärke wird notiert. Diese Messergebnisse finde sich in \autoref{tab:messergebnisse_esr}
%
\begin{table}[H]
    \centering
    \begin{samepage}
        \caption[Messergebnisse ESR]{Messergebnisse zum Versuch zur Elektronen-Spin-Resonanz. Zur jeweiligen Steckspulengröße (klein/mittel/groß) und eingestellten Resonanzfrequenz $f_{\text{res}}$ (in \unit{MHz}) wird die abgelesene Gleichstromstärke $I_{\mathdirectcurrent{}}$ (in \unit{mA}) notiert. Unsicherheiten: $\Delta f = \SI{0.1}{MHz}$, $\Delta I_{\mathdirectcurrent{}} = \SI{0.01}{mA}$.}
        \label{tab:messergebnisse_esr}
        \begin{tblr}{colspec={QS[table-format=3.1]S[table-format=4]}, row{1}={guard}}
            Spule  & $f_{\text{res}}$ / \unit{MHz} & $I_{\mathdirectcurrent{}}$ / \unit{mA} \\
            klein  & 120.0                         & 1133                                   \\
            klein  & 110                           & 1031                                   \\
            klein  & 100                           & 951                                    \\
            klein  & 90                            & 845                                    \\
            klein  & 80.1                          & 757                                    \\
            klein  & 70                            & 666                                    \\
            klein  & 60                            & 567                                    \\
            mittel & 50.1                          & 485                                    \\
            mittel & 40.1                          & 386                                    \\
            mittel & 35.1                          & 338                                    \\
            mittel & 30                            & 290                                    \\
            groß   & 25                            & 242                                    \\
            groß   & 20                            & 168                                    \\
        \end{tblr}
    \end{samepage}
\end{table}
%
Anschließend wurde noch qualitativ die Auswirkung der genauen Position der DPPH-Probe auf die Bildgebung durch das Oszilloskop untersucht.




\section{Auswertung}
\label{sec:auswertung}

\subsection{Elektronenbeugung}
\label{subsec:auswertung_elektronenbeugung}

Wie schon in \autoref{subsec:durchfuehrung_elektronenbeugung} beschrieben, wurde die Beugung von Elektronen an Graphit untersucht. Aus den Beschleunigungsspannungen in \autoref{tab:messergebnisse_beugung} lässt sich die Wellenlänge der Elektronen bestimmen. Die Wellenlänge der Elektronen wird durch mittels \autoref{eq:wellenlaenge_elektronen} aus der vorliegenden Beschleunigungsspannung ermittelt und ergibt sich zu den in \autoref{tab:auswertung_beugung_wellenlaenge} aufgelisteten Werten.
%
\begin{table}[H]
    \centering
    \begin{samepage}
        \caption[Elektronenwellenlängen für gegebene Beschleunigungsspannung]{Elektronenwellenlängen $\lambda_e$ in $\si{\pico\meter}$ bei gegebener Beschleunigungsspannung $U_{\text{B}}$ (in $\si{\kilo\volt}$).}
        \label{tab:auswertung_beugung_wellenlaenge}
        \begin{tblr}{colspec={S[table-format=1.2(1)]S[table-format=2.2(2)]}, row{1}={guard}}
            $U_B$ / \si{\kilo\volt} & $\lambda_e$ / \si{\pico\meter} \\
            2.00(2)                 & 27.42(14)                      \\
            2.50(2)                 & 24.50(10)                      \\
            3.00(2)                 & 22.39(8)                       \\
            3.50(2)                 & 20.73(6)                       \\
            4.00(2)                 & 19.39(5)                       \\
        \end{tblr}
    \end{samepage}
\end{table}
%
Weiter werden aus den gemessenen Abständen in \autoref{tab:messergebnisse_beugung} unter zuhilfenahme von \autoref{eq:bragg} die Abstände der Gitterebenen in Graphit bestimmt.
Dabei werden die Messungen für die beiden Abstände $d_1$ und $d_2$ (siehe \autoref{fig:graphitgitter}) mittels Student-t Verteilung statistisch ausgewertet. Die Gitterabstände ergeben sich so zu:
\begin{align}
    d_1 & = \SI{2.28(19)e-10}{\meter} \\
    d_2 & = \SI{1.3(1)e-10}{\meter}
\end{align}


\subsection{Elektronen im Magnetfeld}
\label{subsec:auswertung_elektronen_b_feld}

Mithilfe der Messwerte aus \autoref{tab:messergebnisse_b_feld} und den Daten zu \autoref{fig:kolben_skizze} kann schließlich der Krümmungsradius $r$ der Elektronenbahn nach der folgenden Formel berechnet werden:
\[r = \sqrt{d^2-s^2} \cdot \frac{d}{s}\]
Hierbei beschreibt $d$ den Durchmesser des Glaskolbens und $s$ die (in \autoref{tab:messergebnisse_b_feld} mit \enquote{$d$} bezeichneten) Auslenkungen von der Nulllage.
Die Krümmungsradien in Abhängigkeit ihrer Auslenkungen sind in \autoref{tab:radien_zu_auslenkungen} aufgelistet.
%
\begin{table}[H]
    \centering
    \begin{samepage}
        \caption[Krümmungsradien zu Auslenkungen]{Berechnete Krümmungsradien $r$ in $\si{\centi\meter}$ zu den Auslenkungen $s$ in $\si{\milli\meter}$ der Elektronen im homogenen Magnetfeld.}
        \label{tab:radien_zu_auslenkungen}
        \begin{tblr}{colspec={S[table-format=2]S[table-format=2.1(2)e1]}, row{1}={guard}}
            $s$ / \unit{mm} & $r$ / \unit{cm} \\
            5               & 1.8(8)e2        \\
            10              & 91(19)          \\
            15              & 60(9)           \\
            20              & 45(5)           \\
            25              & 36(3)           \\
            30              & 30(3)           \\
            35              & 25.1(1.6)       \\
        \end{tblr}
    \end{samepage}
\end{table}
%
Anschließend wurde die magnetischen Induktion aus dem Spulenstrom berechnet. Hierzu bediente man sich der Formel neben \autoref{fig:kolben_skizze} zur Berechnung des $B$-Felds eines Helmholtzspulenpaars. Die magnetische Induktion zwischen dem Helmholzspulenpaar in Abhängigkeit von der diese durchfließende Stromstärke $I$ ist in \autoref{tab:b_feld_zu_stromstaerke} tabelliert.
%
\begin{table}[H]
    \centering
    \begin{samepage}
        \caption[Magnetfeldstärken zu Stromstärken]{Zwischen dem Helmholtzspulenpaar herrschende magnetische Induktion $B$ in $\si{\milli\tesla}$ zu den diese Spulen durchfließenden Stromstärken $I$ in $\si{\milli\ampere}$. Unterschieden wird zusätzlich zwischen den angelegten Beschleunigungsspannungen $U_{\text{HV}}=\{\SI{1.8}{kV}\mathcomma\SI{4}{kV}\}$. Unsicherheiten $\Delta U_{\text{HV}} = \SI{0.02}{kV}$, $\Delta I = \SI{3}{mA}$, ansonsten direkt in der Zelle angegeben.}
        \label{tab:b_feld_zu_stromstaerke}
        \begin{tblr}{colspec={S[table-format=1.1]S[table-format=3(2)]S[table-format=1.2(1)]}, row{1}={guard}}
            $U_{\text{HV}}$ / \unit{kV} & $I$ / \unit{mA} & $B$ / \unit{mT} \\
            1.8                         & 44(9)           & 0.15(3)         \\
            1.8                         & 76(3)           & 0.26(1)         \\
            1.8                         & 104(8)          & 0.36(3)         \\
            1.8                         & 140(17)         & 0.48(6)         \\
            1.8                         & 172(15)         & 0.59(6)         \\
            1.8                         & 202(12)         & 0.69(5)         \\
            1.8                         & 234(12)         & 0.80(4)         \\
            4.0                         & 44(9)           & 0.23(6)         \\
            4.0                         & 76(3)           & 0.38(7)         \\
            4.0                         & 104(8)          & 0.52(9)         \\
            4.0                         & 140(17)         & 0.70(9)         \\
            4.0                         & 172(15)         & 0.87(6)         \\
            4.0                         & 202(12)         & 1.04(8)         \\
            4.0                         & 234(12)         & 1.20(7)         \\
        \end{tblr}
    \end{samepage}
\end{table}
%
Da nun zu bestimmten Auslenkungen respektive Stromstärken sowohl der Krümmungsradius als auch die mangetische Induktion bekannt sind, kann ein Plot des inversen Krümmungsradius $\nicefrac{1}{r}$ gegen $B$ erstellt werden. Es wird wieder zwischen den beiden Beschleunigungsspannungen $U_{\text{HV}}=\{\SI{1.8}{kV}\mathcomma\SI{4}{kV}\}$ unterschieden. Dieser Plot ist auf \autoref{fig:1_over_r_gegen_B} ersichtlich.
%
\begin{figure}[H]
    \centering
    \begin{samepage}
        \includegraphics[width=\linewidth]{../python/plots/kruemmungsradius.pdf}
        \caption[Inverser Krümmungsradius gegen $B$-Feld]{Inverser Krümmungsradius $\nicefrac{1}{r}$ gegen magnetische Induktion $B$}
        \label{fig:1_over_r_gegen_B}
    \end{samepage}
\end{figure}
%
Auf \autoref{fig:1_over_r_gegen_B} wurde darüberhinaus lineare Regressionen über die gezeichneten Kurven gerechnet. Nach \autoref{eq:spezifische_ladung_U} kann die Steigung der Funktion $\nicefrac{1}{r}$ bestimmt werden, indem man auf $\nicefrac{1}{r}$ umformt und nach $B$ ableitet.
\[\pdv{\nicefrac{1}{r}}{B} = \sqrt{\frac{q_{\text{spez}}}{2U}}\]
Formt man diese Steigung, die über die Regression errechnet wurde, wieder auf $q_{\text{spez}}$ um, so erhält man für die spezifische Ladung der Elektronen
\begin{align}
    q_{\text{spez},\SI{1.8}{kV}} & = \\
    q_{\text{spez},\SI{4.0}{kV}} & =
\end{align}


\subsection{Elektronen-Spin-Resonanz}
\label{subsec:auswertung_esr}

Zu den gewählten Resonanzfrequenzen in \autoref{tab:messergebnisse_esr} fließt abhängig von diesen ein gewisser Strom durch die Helmholtzspulen. Dieser Strom kann durch den Zusammenhang:
\[B = \mu_0 \cdot (\nicefrac{4}{5})^{\nicefrac{3}{2}}*\nicefrac{n}{r} \cdot I\]
in eine magnetische Flussdichte umgerechnet werden, welcher in \autoref{tab:esr_b_feld} aufgelistet ist.


\begin{table}[H]
    \centering
    \begin{samepage}
        \caption[Magnetische Flussdichte durch Helmholzspulen]{Magnetische Flussdichte $B$ durch Helmholzspulen in Abhängigkeit der Resonanzfrequenz $f_{\text{res}}$.\\$\Delta f = \SI{0.1}{MHz}$, $\Delta I_{\mathdirectcurrent{}} = \SI{0.01}{mA}$, $\Delta B = \SI{0.1}{mT}$}
        \label{tab:esr_b_feld}
        \begin{tblr}{colspec={QS[table-format=3.1]S[table-format=4]S[table-format=1.1]}, row{1}={guard}}
            Spule  & $f_{\text{res}}$ / \unit{MHz} & $I_{\mathdirectcurrent{}}$ / \unit{mA} & $B$ / \unit{mT} \\
            klein  & 120.0                         & 1133                                   & 4.8             \\
            klein  & 110.0                         & 1031                                   & 4.4             \\
            klein  & 100.0                         & 951                                    & 4.0             \\
            klein  & 90.0                          & 845                                    & 3.6             \\
            klein  & 80.1                          & 757                                    & 3.2             \\
            klein  & 70.0                          & 666                                    & 2.8             \\
            klein  & 60.0                          & 567                                    & 2.4             \\
            mittel & 50.1                          & 485                                    & 2.1             \\
            mittel & 40.1                          & 386                                    & 1.6             \\
            mittel & 35.1                          & 338                                    & 1.4             \\
            mittel & 30.0                          & 290                                    & 1.2             \\
            groß   & 25.0                          & 242                                    & 1.0             \\
            groß   & 20.0                          & 168                                    & 0.7             \\
        \end{tblr}
    \end{samepage}
\end{table}
Trägt man die Werte der magnetischen Flussdichte und die Frequenz gegeneinander auf so erhält man nachdem man die Messwerte linear fittet die in \autoref{fig:auswertung_esr} dargestellte Gerade. Die Steigung gibt nun den Wert des Landé-Faktor $g$ für 1,1-Diphenyl-2-Pikryl-Hydrazyl (DPPH) an, welcher sich zu:
\[g = \num{1.785(7)}\]
ergibt.
\begin{figure}[H]
    \centering
    \begin{samepage}
        \includegraphics[width=0.8\linewidth]{../python/plots/esr.pdf}
        \caption[ESR - Fit]{Fit der Resonanzfrequenz $f_{\text{res}}$ in Abhängigkeit der magnetischen Flussdichte $B$ für DPPH}
        \label{fig:auswertung_esr}
    \end{samepage}
\end{figure}
Laut Literatur würde man einen Wert von $\approx 2$ erwarten, dieser stellt sich jedoch nicht ein und würde sich beispielsweise lediglich bei einem abgeschwächten Magnetfeld einstellen. Dieses abgeschwächte Feld wird in \autoref{tab:esr_b_schwach_feld} aufgelistet, womit ein Landé-Faktor von $g = \num{2.006(8)}$ folgen würde.
\begin{table}[H]
    \centering
    \begin{samepage}
        \caption[Abgeschächtes Magnetfeld]{Abgeschächtes Magnetfeld für nicht planparallele Helmholzspulen in Abhängigkeit der Resonanzfrequenz $f_{\text{res}}$.\\$\Delta f = \SI{0.1}{MHz}$, $\Delta B = \SI{0.1}{mT}$}
        \label{tab:esr_b_schwach_feld}
        \begin{tblr}{colspec={QS[table-format=3.1]S[table-format=1.1]}, row{1}={guard}}
            Spule  & $f_{\text{res}}$ / \unit{MHz} & $B$ / \unit{mT} \\
            klein  & 120.0                         & 4.3             \\
            klein  & 110.0                         & 3.9             \\
            klein  & 100.0                         & 3.6             \\
            klein  & 90.0                          & 3.2             \\
            klein  & 80.1                          & 2.9             \\
            klein  & 70.0                          & 2.5             \\
            klein  & 60.0                          & 2.1             \\
            mittel & 50.1                          & 1.8             \\
            mittel & 40.1                          & 1.5             \\
            mittel & 35.1                          & 1.3             \\
            mittel & 30.0                          & 1.1             \\
            groß   & 25.0                          & 0.9             \\
            groß   & 20.0                          & 0.6             \\
        \end{tblr}
    \end{samepage}
\end{table}

\section{Diskussion}
\label{sec:diskussion}



\section{Zusammenfassung}
\label{sec:zusammenfassung}



\clearpage
% Literaturverzeichnis
\printbibliography

% Abbildungsverzeichnis
\listoffigures

% Tabellenverzeichnis
\listoftables

\end{document}
